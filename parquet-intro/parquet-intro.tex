\documentclass{beamer}

\usepackage{amsfonts}
\usepackage{listings}
\usepackage{adjustbox}
\usepackage{tikz}
\usetikzlibrary{chains,arrows,automata,decorations.markings,positioning,calc,decorations.pathreplacing,patterns}

\setbeamertemplate{footline}[frame number]

\begin{document}

\frame{
    \begin{center}
        \textbf{\huge Parquet file format --- brief introduction}

        \vspace{\baselineskip}

        Wojciech Muła, \href{http://0x80.pl}{0x80.pl} December 2021
    \end{center}

    \vfill

    Major features:
    
    \begin{itemize}
        \item Fixed data schema
        
        \item Columnar storage (by design)

        \item Ability to split multiple rows into "row groups" (file creator decision)

        \item Designed to parallel processing: at the level of row groups and columns

        \item Per-column \& per-row group data compression

        \item Rich data types: numbers, strings, timestamps, bool, arrays, structs, dictionaries, unions, etc.

        \item Uses \textbf{Dremel} encoding to describe nested data structures
    \end{itemize}
}


\frame{
    \frametitle{Algorithm backing the format}

    \begin{itemize}
        
        \item \textbf{Dremel} invented by Google
        
        \item Assigns values only to \texttt{leaves} of schema tree
        
        \item ... inner nodes may be only set to \texttt{null}
        
        \item Allow to repeat any part of schema subtree
        
        \item \ldots but does not support arrays directly
    \end{itemize}

    \vfill
    Sample schema with five columns

    \begin{adjustbox}{max width=7cm}
        \lstinputlisting{schema.txt}
    \end{adjustbox}
}

\frame{
    \frametitle{Column data --- overview}

    Single column data consists \textbf{two} or \textbf{three} arrays of values

    \begin{itemize}
        
        \item Always present \textbf{definition levels} --- which part of path is "defined"
        
        \item Optional \textbf{repetition levels} --- at which part of schema we repeat values
        
        \item \textbf{Actual} data, present if there are not-null values in column

        \item Definition \& repetition levels are sufficient to reconstruct arbitrary record structure
    \end{itemize}
}

\frame{
    \frametitle{Column data --- levels}

    \begin{itemize}
        \item Definition and repetition levels are arrays of unsigned numbers

        \item Their sizes are \texttt{equal} to the number of rows

        \item Repetition level is present only if some part of path may repeats

        \item \ldots this property is set in the data schema
    \end{itemize}
}

\frame{
    \frametitle{Column data --- definition levels}

    Let's assume path \texttt{customer.location.city}. Other possible paths
    are \texttt{customer.location} and \texttt{customer}.
    \vfill

    \begin{adjustbox}{max width=6cm}
        \lstinputlisting{deflevel.txt}
    \end{adjustbox}

    \vfill
    Corresponding JSON

    \begin{adjustbox}{max width=8cm}
        \lstinputlisting{deflevel.json}
    \end{adjustbox}
}

\frame{
    \frametitle{Column data --- definition levels continued}

    \begin{itemize}
        
        \item Definition level equals to "max definition level" means the values is present
        
        \item ... in the example we have only two values in the data array
        
        \item Definition level less than max says at which part of path we set \texttt{null}
        
        \item It's easy to do queries like "SELECT COUNT(*) ... WHERE column IS NOT NULL"
    \end{itemize}
}

\frame{
    \frametitle{Column data --- repetition levels}

    \begin{itemize}
        
        \item similarly to definition level --- decides which part of path repeats
        
        \item if value is defined it means "append" to the tree

        \item ... it doesn't apply to JSON, XML is better
    \end{itemize}

    The same definition levels, but different repetition levels

    \begin{adjustbox}{max width=5cm}
        \lstinputlisting{replevel1.txt}
    \end{adjustbox}

    \begin{adjustbox}{max width=5cm}
        \lstinputlisting{replevel2.txt}
    \end{adjustbox}
}

\frame{
    \frametitle{Parquet format --- part 1}

    \begin{itemize}

        \item Parquet uses the Dremel algorithm underneath
        
        \item It efficiently encodes definition and repetition levels (RLE, compression)
        
        \item ... but exposes them as plain arrays of \texttt{uint16}
        
        \item Parquet uses seven physical types:
            \texttt{boolean},
            \texttt{int32},
            \texttt{int64}h
            \texttt{float32},
            \texttt{float64},
            \texttt{variable-length bytes},
            \texttt{fixed-length bytes}

        \item It may collect some statistics regarding the column, like: nulls count, distinct count, min value, max value

        \item Columnar data is usually compressed
    \end{itemize}
}

\frame{
    \frametitle{Parquet format --- part 2}

    \begin{itemize}
        
        \item Physical types are mapped into logical types, like UTF-8 strings, timestamps, date time, decimal
        
        \item Logical types are also complex ones: structures, arrays, maps, unions, dictionaries
        
        \item Arrays can't be directly represented in Dremel algorithm
        
        \item ... as it lets repeat only "key-value" pairs
        
        \item ... plain arrays are done by adding artificial nodes to schema and then process them in a special way

        \item Unions and dictionaries are data types designed for reduce memory usage
    \end{itemize}
}

\frame{
    \frametitle{Further reading}

    \begin{itemize}
        
        \item \href{https://github.com/apache/parquet-format}{Parquet format} --- a lot of details: data layout, used encodings, design considerations, etc.
        
        \item \href{https://blog.twitter.com/engineering/en_us/a/2013/dremel-made-simple-with-parquet}{Dremel made simple with Parquet} --- detailed overview of interpretation definition and repetition levels
        
        \item \href{https://research.google/pubs/pub36632/}{Dremel: Interactive Analysis of Web-Scale Datasets} -- the original paper
    \end{itemize}
}

\end{document}

\frame{
    \frametitle{}

    \begin{itemize}
        
        \item
        
        \item
        
        \item
        
        \item
        
        \item
        
        \item
        
        \item
    \end{itemize}
}
