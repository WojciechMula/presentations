\documentclass{beamer}

\usepackage{amsfonts}
\usepackage{listings}
\usepackage{adjustbox}
\usepackage{tikz}
\usetikzlibrary{chains,arrows,automata,decorations.markings,positioning,calc,decorations.pathreplacing,patterns}

\setbeamertemplate{footline}[frame number]

\begin{document}

\frame{
    \begin{center}
        \textbf{\huge avo --- "Generate x86 Assembly with Go"}

        \vspace{\baselineskip}

        Wojciech Muła, \href{http://0x80.pl}{0x80.pl} April 2022
    \end{center}

    \vfill

    Project \href{https://github.com/mmcloughlin/avo}{avo} highlights:
    
    \begin{itemize}
        \item A Go program generates assembly files

        \item The generator takes care of register allocation

        \item \ldots\ but we can use explicit hardware registers (for instance \texttt{CX} for shift amounts)

        \item Takes care of proper accesing to structures

        \item \ldots\ with hand-written assembly we need \texttt{go vet -asmdecl}

        \item Can generate Go stub files
    \end{itemize}
}

\frame{
    \frametitle{Major modules --- 1/3}

    \begin{itemize}
        
        \item \texttt{avo/build} --- all instructions, stack allocation, program comments, virtual registers:
        
        \item \ldots\ \texttt{GP64()} --- 64-bit GPR
        
        \item \ldots\ \texttt{GP64().As32()} --- the lower 32 bits of GPR

        \item \ldots\ \texttt{GP64().As16()} --- the lower 16 bits of GPR

        \item \ldots\ \texttt{GP64().As8()} --- the lower 8 bits of GPR

        \item \ldots\ \texttt{GP64().As8H()} --- the higher 8 bits of GPR (like \texttt{AH}, not always available)
        
        \item \ldots\ \texttt{XMM()} --- SSE register

        \item \ldots\ \texttt{YMM()} --- AVX register

        \item \ldots\ \texttt{ZMM()} --- ZMM register
        
        \item \ldots\ \texttt{x := AllocLocal(8)}
    \end{itemize}
}

\frame{
    \frametitle{Major modules --- 2/3}

    \begin{itemize}
        
        \item \texttt{avo/operand} --- labels, immediate types, memory address
        
        \item \ldots\ \texttt{Label("name")} --- declare label
        
        \item \ldots\ \texttt{JMP(LabelRef("name"))} --- jump to label
        
        \item \ldots\ \texttt{ADDQ(U8(42), reg)} --- \texttt{U8}, \texttt{U16}, \texttt{U32}
        
        \item \ldots\ \texttt{Mem\{Base: reg, Index: reg, Scale: imm, Disp: imm\}} --- description of x86 address
    \end{itemize}
}

\frame{
    \frametitle{Major modules --- 3/3}

    \begin{itemize}
        
        \item \texttt{avo/buildtags} --- construct build tags, like \texttt{go:build !appengine \&\& !noasm \&\& gc \&\& !noasm}
        
        \item \texttt{avo/reg} --- types for registers, names of physical registers, etc.
        
    \end{itemize}
}

\frame{
    \frametitle{Example 1 --- implicit registers}

    \begin{adjustbox}{max width=8cm}
    \lstinputlisting[language=Go,numbers=left]{1.go}
    \end{adjustbox}
}

\frame{
    \frametitle{Example 1 --- output}

    \lstinputlisting[numbers=left]{1.s}
}

\frame{
    \frametitle{Example 2 --- explicit registers}

    \begin{adjustbox}{max width=8cm}
    \lstinputlisting[language=Go,numbers=left]{2.go}
    \end{adjustbox}
}

\frame{
    \frametitle{Example 2 --- output}

    \lstinputlisting[numbers=left]{2.s}
}

\frame{
    \frametitle{Example 3 --- explicit registers}

    \begin{adjustbox}{max width=6.5cm}
    \lstinputlisting[language=Go,numbers=left]{3.go}
    \end{adjustbox}
}

\frame{
    \frametitle{Example 3 --- output}

    \lstinputlisting[numbers=left]{3.s}
}

\end{document}

\frame{
    \frametitle{}

    \begin{itemize}
        
        \item
        
        \item
        
        \item
        
        \item
        
        \item
        
        \item
        
        \item
    \end{itemize}
}
